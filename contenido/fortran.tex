\chapter{Conceptos de Fortran 90}

\begin{chapquote}{El Erick, 2024}
    ``Tú no usas Fortran, ¿verdad? Tú sí eres inteligente.''
\end{chapquote}

\section{Fortran apesta}

\begin{itemize}
	\item{El lenguaje no distingue mayúsculas de minúsculas.}
	\item{Por defecto, las variables que
			\begin{itemize}
				\item{inician con $i$, $j$, $k$, $l$, $m$ son enteras}
				\item{y el resto de letras son reales.}
			\end{itemize}
		}
	\item{El código que escribamos siempre tendrá estructura modular y
			jerárquica:
			\begin{itemize}
				\item{Programas}
				\item{Subrutinas}
				\item{Funciones}
			\end{itemize}
		}
	\item{El código siempre se ejecutará de izquierda a derecha y de arriba
		a abajo.}
\end{itemize}

\section{Sintaxis del lenguaje}

\begin{itemize}
	\item{\texttt{IMPLICIT NONE}: elimina la asignación por defecto de los enteros. Causa errores y advertencias cuando una variable no sea usada}
	\item{\texttt{OPEN(UNIT, FILE, STATUS, ERR)}: }
\end{itemize}
