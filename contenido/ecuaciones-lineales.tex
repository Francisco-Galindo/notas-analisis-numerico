\chapter{Solución de ecuaciones lineales}

Normalmente, cuando se resuelven problemas en ingeniería, es normal encontrar
sistemas de ecuaciones simultáneas:

\begin{align*}
    f_1(x_1, x_2,\ \dots \ , x_n) &= 0 \\
    f_2(x_1, x_2,\ \dots \ , x_n) &= 0 \\
    \vdots \\
    f_n(x_1, x_2,\ \dots \ , x_n) &= 0
\end{align*}

\section{Solución de sistemas de ecuaciones lineales}

Los sistemas de ecuaciones lineales de orden $n \leq 3$ son más fáciles de
resolver mediante métodos como el de Cramer, que consiste en resolver el
siguiente sistema mediante determinantes:

\begin{equation*}
    \begin{aligned}[c]
        a_{11} x + a_{12} y + a_{13} z = b_1 \\
        a_{21} x + a_{22} y + a_{23} z = b_2 \\
        a_{31} x + a_{32} y + a_{33} z = b_4 \\
    \end{aligned}
    \ \Longrightarrow
    \begin{aligned}[c]
        \begin{pmatrix}
            a_{11} & a_{12} & a_{13}\\
            a_{21} & a_{22} & a_{23}\\
            a_{31} & a_{32} & a_{33}\\
        \end{pmatrix}
        \begin{pmatrix}
            x \\
            y \\
            z
        \end{pmatrix}
        =
        \begin{pmatrix}
            b_1 \\
            b_2 \\
            b_3
        \end{pmatrix}
    \end{aligned}
\end{equation*}

Con soluciones
\begin{equation*}
    x = \frac{
    \begin{vmatrix}
        b_1 & a_{12} & a_{13}\\
        b_2 & a_{22} & a_{23}\\
        b_3 & a_{32} & a_{33}\\
    \end{vmatrix}}{
    \begin{vmatrix}
        a_{11} & a_{12} & a_{13}\\
        a_{21} & a_{22} & a_{23}\\
        a_{31} & a_{32} & a_{33}\\
    \end{vmatrix}},
    y = \frac{
    \begin{vmatrix}
        a_{11} & b_1 & a_{13}\\
        a_{21} & b_2 & a_{23}\\
        a_{31} & b_3 & a_{33}\\
    \end{vmatrix}}{
    \begin{vmatrix}
        a_{11} & a_{12} & a_{13}\\
        a_{21} & a_{22} & a_{23}\\
        a_{31} & a_{32} & a_{33}\\
    \end{vmatrix}},
    z = \frac{
    \begin{vmatrix}
        a_{11} & a_{12} & b_1\\
        a_{21} & a_{22} & b_2\\
        a_{31} & a_{32} & b_3\\
    \end{vmatrix}}{
    \begin{vmatrix}
        a_{11} & a_{12} & a_{13}\\
        a_{21} & a_{22} & a_{23}\\
        a_{31} & a_{32} & a_{33}\\
    \end{vmatrix}}
\end{equation*}

Para resolver un sistema de ecuaciones $3 \times 3$, es necesario calcular 4
determinantes. El número de multiplicaciones necesarias para calcular un
determinante de una matriz $A \in M_{n \times n}$ se puede calcular con la
ecuación de recurrencia
\[ y_{n+1} = (n+1)(y_n + 1), \]

\noindent donde $y_n$ es el número de multiplicaciones necesarias durante el
cálculo del determimante de la matriz. Al resolver la ecuación de recurrencia,
se obtiene la solución directa
\[
    y(n) = \sum_{k=1}^{n-1} \frac{n!}{k!} = n! \sum_{k=1}^{n-1} \frac{1}{k!}
\]

Este es un valor que crece extremadamente rápido, por ejemplo, una matriz de
orden 4 necesita 40 multiplicaciones y una matriz de orden 5 necesita 200. El
crecimiento en el número de operaciones necesarias crece de manera exponencial,
lo que hace que se vuelva altamente impráctico para matrices de orden grande.
Dado que el método de Cramer se basa en el cálculo de determinantes, el método
se vuelve inútil cuando se quieren resolver sistemas de ecuaciones grandes.

Debido a lo anterior es que, al momento de resolver sistemas de ecuaciones
grandes, se utilizará el método de \emph{eliminación gaussiana}. Otros métodos,
como el de \emph{descomposición LU}, empleado para factorizar matrices, están
también basados en la eliminación gaussiana.

\section{Eliminación Gaussiana}

La eliminación de renglones mediante operaciones elementales funciona para
matrices de cualquier orden, así que se antoja automatizar el proceso para
resolver cualquier sistema de ecuaciones lineales mediante un programa de
computadora. El esquema más sencillo es la \emph{eliminación gaussiana simple}.

El método tiene todas las características de la resolución de sistemas de
ecuaciones lineales. Se le llama simple porque no toma en cuenta ciertas
condiciones, como aquella donde todo un renglón se llena de ceros, dando como
resultado que el sistema es compatible indeterminado.

Considere el siguiente sistema de ecuaciones:

\[ \spalignsys {
    a_{11}x_1 + a_{12}x_2 + a_{13}x_3 + \dots + a_{1n}x_n = b_1;
    a_{21}x_1 + a_{22}x_2 + a_{23}x_3 + \dots + a_{2n}x_n = b_2;
    a_{31}x_1 + a_{32}x_2 + a_{33}x_3 + \dots + a_{3n}x_n = b_3;
    \.       \+          \+          \+ \vdots \+;
    a_{n1}x_1 + a_{n2}x_2 + a_{n3}x_3 + \dots + a_{nn}x_n = b_n}
\]

% Hablar sobre sustitución hacia atrás y hacia adelante

Mediante operaciones elementales, puede modificarse el sistema, escalando una de
las filas y sumándola

\paragraph*{Sustitución hacia atrás:} De la matriz triangular resultante, se
puede inferir del último renglón que:
\[
\]

\begin{ex}
    
    Emplee la eliminación de Gauss simple para resolver
    \[ \spalignsys{
        3x_1 \+ -0.1x_2 \+ -0.2x_3 = 7.85;
        0.1x_1 \+ 7x_2 \+ -0.3x_3 = -19.3;
        0.3 \+ -0.2x_2 \+ 10x_3 = 71.4}
    \]

\end{ex}
