\chapter{Raíces de ecuaciones}

\begin{chapquote}{Lozano, 2024}
    ``No se preocupen si no entienden esto, yo tampoco.''
\end{chapquote}


\section{Antecedentes matemáticos}

En general, existen dos tipos de funciones matemáticas:

\begin{definition}[Función algebraica]
	En el caso de una variable, se dice que una función \(f(x)\) es
	algebraica si se expresa de la siguiente manera:

	\[
		f_n y^n + f_{n-1} y^{n-1} + ... + f_1 y + f_0 = 0
	\]

	Donde cada \(f_i (i = 0, 1, ..., n)\) es un polinomio de la forma:

	\[
		f_i = a_{in} X^i + a_{i(n-1)} x^{i-1} + ... + a_{i1} x + a_{i0}
	\]
\end{definition}

\begin{definition}[Función trascendental]
	Es aquella que no es algebraica.
\end{definition}

\begin{eg}
	Las funciones trigonométricas, exponenciales, logarítmicas, las de
	Bessel, Jacobi, Mittag Leffler, entre muchas otras, son trascendentes.
	Por ejemplo, el logaritmo natural se puede definir así:

	\[
		\boxed{\ln(x) = \int_1^x \frac{d \tau}{\tau}, x > 0}
	\]

\end{eg}

Solucionar ecuaciones algebraicas o trascendentes implica el uso de dos tipos de
métodos numéricos: los métodos cerrados y los métodos abiertos. Los primeros
utilizan dos valores que encierran a la raíz, mientras que los segundos sólo
uno.

\section{Métodos cerrados}

Este tipo de métodos aprovecha el cambio de signo que se presenta cuando una
función pasa a través del eje de las abscisas para detectar una raíz. Se les
conoce por ese nombre porque necesitan un intervalo cerrado que contenga a la
raíz. Un método cerrado muy sencillo es el \textbf{método de bisección}.

\subsection{Método de bisección}

En general, si $f(x)$ es una función real y continua en el intervalo $[x_l,
x_u]$ y además $f(x_l)$ y $f(x_u)$ tienen signos opuestos, \textit{i.e.}
$f(x_l) f(xu) < 0$, se sabe que \textbf{existe al menos una raíz entre $x_l$ y
$x_u$}.

El método consiste en dividir el intervalo original a la mitad (hacer una
bisección) y descartar aquella mitad donde no haya raíces. El proceso se repite
tantas veces como sea necesario hasta que el intervalo sea, en términos
prácticos, un sólo punto: la raíz buscada. Es un \textit{binary search de
raíces}.

\subsubsection{Algoritmo para el método de bisección}

\begin{enumerate}
	
	\item Elija el intervalo que encierra la raíz, es decir, $[x_l, x_u]$.
		Verifique que $f(x_l)f(x_u) < 0$.

	\item Obtenga la aproximación de la raíz con $x_r \approx \frac{x_l +
		x_u}{2}$

	\item Para determinar el nuevo sub-intervalo, evalúe:
		\begin{itemize}
			\item Si $f(x_l)f(x_r) < 0$, entonces la raíz debe
				encontrarse en el intervalo $[x_l, x_r]$. El
				algoritmo se seguirá ejecutando de manera
				recursiva.
			\item Si $f(x_l)f(x_r) > 0$, entonces la raíz debe
				encontrarse en el intervalo $[x_r, x_u]$. El
				algoritmo se seguirá ejecutando de manera
				recursiva.

			\item Si $f(x_l) f(x_r) = 0$, entonces $x_r$ es la raíz
				buscada, o
				\[
					\varepsilon_a < \varepsilon_s
				\]

				Donde $\varepsilon_a$ es el \textbf{error relativo
				aproximado} y $\varepsilon_s$ es la
				\textbf{tolerancia} (\textit{sufferance})
				prefijada.
		\end{itemize}
\end{enumerate}

Pocas veces se conoce a priori el valor de la raíz real. Por ello, se
debe definir una aproximación mediante el error relativo aproximado:

\[
	\varepsilon_a = \left| \frac{x_i - x_{i-1}}{x_i} \right| \times 100\%
\]

Una fórmula comúnmente usada para la elegir una tolerancia en cálculos numéricos
donde se quiere garantizar un número $n$ de cifras significativas es la
siguiente:

\[
	\varepsilon_s = 0.5 \times 10^{2-n} \%
\]

\begin{ex}

	Encuentre la raíz de la ecuación \ref{primer-problema}:

	\[
		x^2 - 4 - sin(x) = 0
	\]

	\begin{solution}
		Tomando $f(x) = x^2 - 4 - sin(x)$, uno puede utilizar un
		lenguaje de programación para implementarel algoritmo ya
		descrito.

		\lstinputlisting[language=Fortran, caption=Implementación del
		método de bisección en Fotran,
		label={lst:bisect-fortran}]{./programas/metodo-biseccion/biseccion.f90}
	\end{solution}
	
\end{ex}
